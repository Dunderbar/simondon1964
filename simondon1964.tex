%----------------------------------------
% Preamble
%----------------------------------------

\documentclass[a4paper]{article}
\usepackage[utf8]{inputenc}
\usepackage[colorlinks = true,
            linkcolor = blue,
            urlcolor  = blue,
            citecolor = blue,
            anchorcolor = blue]
            {hyperref}
\begin{document}

%----------------------------------------
% Title Page
%----------------------------------------

\Large
\begin{center}
Foundations of the Hylemorphic Model:\\
Technology of the Capture of Form\\

\hspace{10pt}

% Author name
\large
Gilbert Simondon\\
\large
1964\\

\hspace{10pt}

% Other collaborators
\small
Translated by Taylor Adkins,\\
\href{https://fractalontology.wordpress.com}{fractalontology.wordpress.com}\\
(\href{https://twitter.com/tadkins613}{@tadkins613})\\
[1\baselineskip]
Republished by Alexander Longworth-Dunbar,\\
University of Manchester\\
\href{mailto:alexander.longworth-dunbar@student.manchester.ac.uk}{alexander.longworth-dunbar@student.manchester.ac.uk}\\
(\href{https:\\twitter.com/@alexldunbar}{@alexldunbar})


\end{center}

\hspace{10pt}

\normalsize

This document is a republishing of Taylor Adkins' 2007 translation of the first chapter (pp. 27-50) of Gilbert Simondon's original 1964 text \textit{L’individu et sa Genèse Physico-Biologique} (The Physico-Biological Genesis of the Individual), published by Presses Universitaires de France in 1964. The original translation can be found in two parts on the website Fractal Ontology, which Adkins runs alongside Joseph Weissman. In the event that anything happens to the website or the original pages containing the translation I have preserved both parts using the \href{https://perma.cc}{perma.cc} archiving service provided by the \href{http://lil.law.harvard.edu/}{Harvard Library Innovation Lab}:\\

Part 1: \href{https://perma.cc/CSJ6-YQ6Q}{perma.cc/CSJ6-YQ6Q}

Part 2: \href{https://perma.cc/CPD3-FNVK}{perma.cc/CPD3-FNVK}\\

As the more than forty years before the original French publication of Simondon's text and this tentative first translation as well as the continued absence of a full translation of this work more than a decade later will attest, Simondon's work, though undeniable in its powerful influence on late twentieth century French philosophy (most notably through Gilles Deleuze, Bernard Stiegler, and Francois Laruelle, to name but a few), is still yet to receive the proper attention it deserves from Anglophone scholarship.

By republishing this translation I am hoping to contribute somewhat towards making this highly influential thinker's work more readily accessible to an English-speaking audience, such that his profoundly influential ideas might be more widely grasped and utilised and, furthermore, in the hope that Simondon might eventually be given the proper recognition he deserves as one of the most important philosophers of the last century.

\vfill

\begin{flushright}
Alexander Longworth-Dunbar\\
University of Manchester\\
August 2017\\
\end{flushright}

%----------------------------------------
% How to Cite
%----------------------------------------

\newpage
\begin{center}
{\noindent{\Large How to Cite}}
\end{center}
Original French publication:\\
[1\baselineskip]
Simondon, Gilbert, \textit{L’individu et sa Genèse Physico-Biologique} (Paris: Presses Universitaires de France, 1964).\\
[2\baselineskip]
Original translations by Taylor Adkins\footnote{\ It is recommended here to include archived copies of the translations available at the provided perma.cc links alongside the original URLs to protect against any future instances of link rot.}:\\
[1\baselineskip]
Adkins, Taylor, \lq{}Translation: Simondon and the Physico-Biological Genesis of the Individual\rq{}, \textit{Fractal Ontology}, 3 October 2007\\ $\langle$\href{https://fractalontology.wordpress.com/2007/10/03/translation-simondon-and-the-physico-biological-genesis-of-the-individual/}{https://fractalontology.wordpress.com/2007/10/03/translation-simondon-and-the-physico-biological-genesis-of-the-individual/}$\rangle$ [accessed 3 August 2017],\\
archived at $\langle$\href{https://perma.cc/CSJ6-YQ6Q}{https://perma.cc/CSJ6-YQ6Q}$\rangle$ [captured 2 August 2017].\footnote{\ This first reference covers section one (The Conditions of Individuation) of this republication.}\\
[1\baselineskip]
Adkins, Taylor, \lq{}Translation: Simondon, Completion of Section I, Chapter 1, The Individual and Its Physico-Biological Genesis\rq{}, \textit{Fractal Ontology}, 19 October 2007\\
$\langle$\href{https://fractalontology.wordpress.com/2007/10/19/translation-simondon-completion-of-section-i-chapter-1-the-individual-and-its-physico-biological-genesis/}{https://fractalontology.wordpress.com/2007/10/19/translation-simondon-completion-of-section-i-chapter-1-the-individual-and-its-physico-biological-genesis/}$\rangle$ [accessed 3 August 2017],\\
archived at $\langle$\href{https://perma.cc/CPD3-FNVK}{https://perma.cc/CPD3-FNVK}$\rangle$ [captured 2 August 2017].\footnote{\ This second reference covers sections two (The Validity of the Hylemorphic Model) and three (The Limits of the Hylemorphic Model) of this republication.}\\
[2\baselineskip]
This republication:\\
[1\baselineskip]
Longworth-Dunbar, Alexander, \lq{}Gilbert Simondon: Foundations of the Hylemorphic Model\rq{}, \textit{Academia.edu}\\
$\langle$\href{https://academia.edu/placeholder}{https://academia.edu/placeholder}$\rangle$ [date accessed ...].\footnote{\ If referencing this republication it would be recommended to further reference Adkins' translation and/or Simondon's original text.}

\newpage
\tableofcontents{}

%----------------------------------------
% Section 1
%----------------------------------------

\newpage
\section{The Conditions of Individuation}

The notions of form and matter can help solve the problem of individuation only if they are first compared to its position. So if by the contrast it was discovered that the hylemorphic system expresses and contains the problem of individuation, it would be necessary, under pain of locking ourselves [sous peine de s’enfermer] into begging the question, to regard the research of the principle of individuation as logically anterior to the definition of matter and form.

It is difficult to consider the notions of form and matter as innate ideas. However, at the moment when one would be tempted to assign a technological origin to them, one is arrested by the remarkable capacity of generalization which these notions have. It is not only the clay and the brick, the marble and the statue which can be thought according to the hylemorphic model, but also a great number of formal, genetic and compositional actualities, in the living world and the psychic domain. The logical force of this model is such that Aristotle could use it to support a universal system of classification which applies itself to reality by following a logical path as well as a physical path, by ensuring the agreement of the logical order and the physical order, and by authorizing inductive knowledge. Even the ratio of the soul and the body can be thought according to the hylemorphic model.


A base as narrow as that of the technological operation hardly appears able to support a paradigm having a similar force of universality. It is thus appropriate to examine the base of the hylemorphic model, to appreciate the direction and the range of the role played in its genesis by the technical experiment.

The technological character of the origin of a model does not invalidate this model, with the condition that the operation which is used as a basis for the formation of the utilized concepts passes entirely and expresses itself without deterioration in the abstract model. If, on the contrary, the abstraction is carried out in an unfaithful and summary manner, by masking one of the fundamental dynamisms of the technical operation, the model is false. Instead of having a true paradigmatic value, it is nothing more than a comparison, a more or less rigorous juxtaposition according to the cases.

However, in the technical operation which gives rise to an object having form and matter, like a clay brick, the real dynamism of the operation is extremely far from being able to be represented by the matter-form couple. The form and the matter of the hylemorphic model are an abstract form and an abstract matter. The definite being that one can show, this brick drying on this board, does not result from the union of an unspecified matter and an arbitrary form. If one takes fine sand, that it is wet and then one puts it in a brick mould: with the release from the mould, one will obtain a sand heap and not a brick. If one takes clay and then passes it to the rolling mill or the spinneret: one will obtain neither plate nor wire, but an accumulation of broken layers and course cylindrical segments. The clay, conceived as supporting an indefinite plasticity, is the abstract matter. The right-angled parallelepiped, conceived as a brick form, is an abstract form. The concrete brick does not result from the union of the clay’s plasticity and the parallelepiped. So that there can be a parallelepipedic brick, a really existing individual, it is necessary that an effective technical operation institutes a mediation between a given clay mass and this notion of the parallelepiped. However, the technical operation of molding does not itself suffice. Moreover, it does not institute a direct mediation between a given mass of clay and the abstract form of the parallelepiped\footnote{\ I.e. between the reality of an order of magnitude higher than future the individual, concealing the energy conditions of the molding, and the reality-matter, which is, by grain, in its availability, of an order of magnitude lower than that of the future individual, the real brick.}; the mediation is prepared by two chains of preliminary operations which make matter and form converge toward a common operation. To give a form to clay is not to impose the parallelepiped form on rough clay: it is to pack prepared clay in a manufactured mold. If one divides the two ends of the technological chains, the parallelepiped and the clay in the quarry, one tests the impression of realizing, in the technical operation, an encounter between two realities of heterogeneous domains, and institutes a mediation, by communication, between an inter-elementary order, macro-physical, larger than the individual, and an intra-elementary order, micro-physical, smaller than the individual.

Precisely, in the technical operation, it is the mediation itself which should be considered: it consists, in the chosen case, in making a prepared block of clay fill without void a mold and, after release from the mold, dry while preserving without cracks or damage this definite contour. However, the preparation of clay and the construction of the mold are already an active mediation between rough clay and the imposable geometrical form. The mold is constructed so as to be able to be opened and closed without damaging its contents. Certain shapes of solids, geometrically conceivable, became realizable only with very complex and subtle artifices. The art of constructing the mold is, nowadays still, one of the most delicate aspects of the foundry. The mold, moreover, is not only built; it is also prepared: a defined coating, a dry powdering that will prevent wet clay from adhering to the walls at the time of the release from the mold, by disaggregating it or by forming cracks. To give a form, it is necessary to construct such a defined mold, prepared in such a fashion, with such a species of material. There thus exists a first advance which goes from the geometrical form to the concrete material mold, parallel with the clay, existing in the same way as it, posed by the dimensions of it, does in an order of magnitude that is easily formalized. As for clay, it is also subjected to a preparation; as a raw material, it is what the shovel raises to the surface at the edge of the marsh, with roots of rush, and gravel grains. Dried, crushed, sifted, shaped, lengthily kneaded, it becomes this homogeneous and consistent dough having a rather great plasticity to be able to embrace the contours of the mold in which one presses it, and firm enough to preserve this contour during the time necessary for that plasticity to disappear. In addition to the purification, the preparation of rough clay has a goal of obtaining homogeneity and a more appropriate degree of humidity for reconciling plasticity and consistency. There is in the rough clay an aptitude for becoming a plastic mass with the dimensions of a future brick because of the colloidal properties of aluminium hydro-silicates: these are those colloidal properties that effectuate the action of the technical half-chain bordering on the prepared clay; the molecular reality of clay, and the water which it absorbs, organizes itself by the preparation so as to be able to direct itself during the individuation as a homogeneous totality to the stage of the brick in a train of appearances. The prepared clay is that through which each molecule will effectively be put in communication, whatever its place compared to the walls of the mold, with the ensemble of pressure exerted by these walls. Each molecule intervenes on the level of the future individual, and thus enters in an interactive communication with the order of magnitude superior to the individual.

From its dimension, the other technical half-chain descends towards the future individual; the parallelepipedic form is not any form; it already contains a certain schematic which can direct the construction of the mold, which is an ensemble of coherent operations contained in an implicit state; clay is not only passively deformable; it is actively plastic, because it is colloidal; its faculty of receiving a form is not distinguished from that of keeping it, because to receive and to keep are the same: to undergo a deformation flawlessly and with coherence of the molecular chains. The preparation of clay is the constitution of this state of equal distribution of the molecules, of this arrangement in chains; the setting into form is already commenced at the time when the craftsman stirs the paste before introducing it into the mold. Because the form is not only the fact of being parallelepipedic; it is also the fact of being without cracks in the parallelepiped, without bubbles of air, split: fine cohesion is the result of a formalization; and this setting into form is only the exploitation of the colloidal characters of this elementary form without which nothing would be possible, and which is homogeneous compared to the form of the mold: there is only, in the two technical half-chains, a change of scale. In the swamp, clay has its colloidal properties as well, but they are there molecule by molecule, or grain by grain; this is prior to the form, and it is what later will maintain the homogeneous and well molded brick. The quality of the matter is the form’s origin, the form’s element which the technical operation forces to change its scale. In the other technical half-chain, the geometrical form concretizes itself, becomes a dimension of the mold, collected wood, sawdust or damp wood\footnote{\ The mold, thus, is not only the mold, but the technical term of the inter-elementary chains, which comprise vast sets locking up the future individual (working, workshop, press, clay) and containing potential energy. The mold totalizes and accumulates these inter-elementary relations, as prepared clay totalizes and accumulates the molecular inter-elementary interactions of aluminium hydro-silicates.}.

The technical operation prepares two half-chains of transformations that meet at a certain point, when the two created objects are compatible, are on the same scale; this comparison is not single and unconditional; it can be done through stages; what one considers as single formalization is often only the last episode of a series of transformations; when the block of clay receives the final deformation which enables it to fill the mold, its molecules are not reorganized completely and in one move; they displace a little the ones compared to the others; their topology is maintained, it only acts as the latest total deformation. However, this total deformation is not only a formalization of clay by its contour. Clay gives a brick because this deformation operated on masses in which the molecules are already arranged the ones compared to the others, without air, without sand grains, with a good colloidal equilibrium; if the mold did not control in a recent deformation all these pre-established former arrangements, it would not give any form; one can say that the shape of the mold operates only on the shape of clay, not on the clay matter. The mold limits and stabilizes rather than only imposing a form: it gives the end of the deformation and achieves it by stopping it according to a definite contour: it modulates the ensemble of the already formed networks: the gesture of the workman who fills the mold and compresses the clay continues the former gesture of kneading, stretching, shaping: the mold plays the part of a fixed set of modeling hands, acting like arrested forming hands. One could make a brick without a mold, with one’s hands, prolonging the shaping by a fashioning that would continue it without rupture.

Matter is matter because it receives a positive property which enables it to be modeled. To be modeled does not mean to undergo arbitrary displacements, but to order its plasticity according to definite forces that stabilize the deformation. The technical operation is the mediation between an inter-elementary unit and an infra-elementary unit. The pure form already contains gestures, and the primary matter has the capacity to become; the gestures contained in the form meet the becoming of the matter and modulate it. So that the matter can be modulated in its becoming, it is necessary that it is, like the clay at the time when the workman presses in the mold, of a deformable reality, i.e. of the reality which does not have a definite form, but all the indefinite forms, dynamically, because this reality, at the same time that it possesses inertia and consistency, is the agent of force, at least for a moment, and is identified point by point with this force; so that the clay fills the mold, it is not enough that it is plastic: it is necessary that it transmits the pressure that the workman presses on it, and that each point of its mass is a center of forces; clay is pushed in the mold which it fills; it propagates with it in its mass the energy of the workman. During the time of the filling, a potential energy actualizes itself.\footnote{\ This energy expresses the macroscopic state of the system containing the future individual; it is of an inter-elementary origin; however, it enters into interactive communication with each molecule of the matter, and it is by this communication that the form emerges, contemporary with the individual.} It is necessary that the energy which pushes the clay exists, in the system mold-hand-clay, in potential form, so that the clay fills all empty space, being developed in any direction, arrested only by the edges of the mold. The walls of the mold intervene then not simply as the materialized geometrical structures, but point by point as fixed places which do net let the expanding clay advance and oppose to the pressure only a developed equal force in the contrary direction (principle of reaction), without carrying out any work, since they are not displaced.

The walls of the mold play compared to a clay element the same part as an element of this clay compared to another close element: the pressure of an element compared to another within the mass is almost as strong as that of an element of the wall compared to an element of the mass; the only difference resides in this fact that the wall does not displace, whereas the elements of clay can displace the ones compared to the others and to the wall.\footnote{\ Thus the individual constitutes itself by this act of communication, within a society of particles in reciprocal interaction, between all the molecules and the action of molding.} A potential energy being translated within clay by compressive forces actualizes itself during the filling. The vehicular matter with potential energy actualizes itself; the form, represented here by the mold, plays an informative part by exerting forces without work, forces that limit the actualization of the potential energy which the matter is momentarily carrying. This energy can, indeed, actualize itself according to such or such direction, with such or such speed: the form is the limit. The relation between matter and form is not thus made between inert matter and form coming from the outside: there is operation common to and at the same level of existence between matter and form; this common level of existence, is that of force, coming from an energy momentarily transported by the matter, but extracted from a state of the total inter-elementary system of a superior dimension, and expressing individuated limitations. The technical operation constitutes two half-chains which, starting from the raw material and from the pure form, come into contact with one other and reunite. This union is made possible by the dimensional congruence of the two ends of the chains; the successive links of development transfer characters without inventing new ones: they establish only changes of order of magnitude, levels, and of state (for example the passage of the molecular state to a molar state, of the dry state to a wet state); what there is at the end of the material half-chains, is the aptitude of the matter for transporting point by point a potential energy which can cause a movement in an indeterminate direction; what there is at the end of the formal half-chains, is the aptitude of a structure to condition a movement without achieving a work, by a play of forces that do not displace their point of application.This assertion is not rigorously true however; so that the mold can limit the expansion of the plastic dough and direct this expansion statically, it is necessary that the walls of the mold develop a force of reaction equal to the pressure of the clay; the clay ebbs and squashes itself, filling the voids, when the reaction of the walls of the mold is slightly more elevated than the forces which are exerted in other directions at the interior of the clay mass; when the mold is filled completely, on the contrary, the internal pressures are everywhere equal with the forces of the reaction of the walls, so that any movement becomes impossible. The reaction of the walls is thus the static force which directs clay during the filling, by prohibiting the expansion according to certain directions. However, the forces of reaction can only exist in consequence of one very small elastic inflection of the walls; one can say that, from the point of view of matter, the formal wall is the limit from which a displacement in a determinate direction is only possible at the price of a very large increase in work; but so that this condition of the increase in work is effective, it is necessary that it starts to be realized, before the equilibrium merely breaks and so that the matter does not take other directions in which it is not limited, pushed by the energy that it transports with it and self actualizes while advancing; thus it is necessary that there is a light work from the walls of the mold, that which corresponds to the weak displacement of the point of application of the forces of reaction. But this work is not added to that of the actualization of transported energy produced by the clay; it is not cut off any either: it does not interfere with it; it can moreover be reduced as needed; a mold out of thin wood deforms notably under the abrupt pressure of clay, then returns gradually in place; a mold out of thick wood displaces less; a cast iron or flint mold displaces extremely little.

Moreover, the positive work of re-installation compensates for the mainly negative of deformation. The mold can have some elasticity; it must not be plastic. It is as forces that matter and form are put in presence. The only difference between the mode of these forces for the matter and the form reside in that the forces of the matter come from a transported energy by the matter and are always available, while the forces of the form are forces which produce only a slight amount of work and intervene as the limits of the actualization of the energy of the matter. It is not in the infinitely short moment, but in becoming it, that form and matter differ; the form is not the vehicle of potential energy; the matter is in-formable matter only because it can be point by point the vehicle of an energy which actualizes itself\footnote{\ Although this energy is an energy of state, an energy of the inter-elementary system; it is in this interaction of two orders of magnitude, on the level of the individual as it encounters forces through which the communication between orders of magnitude consists, under the aegis of a singularity, principle of form, that individuation starts. The mediating singularity is the mold here; in other cases, in Nature, it can be the stone which starts the dune, the gravel which is the germ of an island in a drifting river of alluvia: it is of the intermediate level between inter-elementary dimensions and infra-elementary dimensions.}; the preliminary treatment of the raw material has as a function to render the support matter homogeneous from a definite potential energy; it is by this potential energy that the matter becomes; the form, it does not become. In the instantaneous operation, the forces which are those of the matter and the forces which come from the form do not differ; they are homogeneous the ones compared to the others and form part of the same instantaneous physical system; but they do not form part of the same temporal unit. The work exerted by the forces of elastic deformation of the mold is nothing anymore after the molding; the forces were cancelled, or were degraded in heat, and did not produce anything on the order of magnitude of the mold. On the contrary, the potential energy of the matter is actualized on an order greater than the clay mass by giving a distribution of the elementary masses. For this reason the preliminary treatment of the clay prepares this actualization: it makes the molecule interdependent of the others molecules and the deformable unit, so that each piece also takes part in the potential energy whose actualization is the molding; it is essential that all the pieces, without discontinuity or privilege, have the same chances of deforming in any direction; a clot, a stone, come within provinces of non-participation to this potentiality which actualizes itself by locating its support: they are parasitic singularities. The fact that there is a mold, i.e. limits of actualization, created in the matter a state of reciprocity of the forces leading to equilibrium; the mold does not act on the outside by imposing molecule on molecule, piece by piece; the clay, at the end of the molding, is the mass in which all the forces of deformation meet in all the directions of the equal forces and contrary direction which founds its equilibrium. The mold translates its existence within the matter by making it tend towards a condition of equilibrium. So because this equilibrium exists it is necessary that at the end of the operation there remains a certain quantity of potential energy still unactualized, contained in the whole system. It would not be exact to say that the form plays a static part whereas the matter plays a dynamic part; in fact, so that there is a single system of forces, it is necessary that matter and form both play a dynamic role; but this dynamic equality is only true in that moment. The form does not evolve or modify itself, because it receives no potentiality, whereas the matter evolves. It carries uniformly widespread potentialities and sets out again in it; the homogeneity of the matter is the homogeneity of its possible becoming. Each point has as many chances as all the others; the matter taking form is in a state of complete internal resonance; what occurs in a point redounds on all the others, the becoming of each molecule retains itself in the becoming of all the others in all the points and in all directions; the matter is that whose elements are not isolated the ones from the others nor heterogeneous the ones compared to the others; any heterogeneity is a condition of the non-transmission of the forces, therefore of internal non-resonance. The plasticity of clay is the capacity to be in a state of internal resonance as soon as it is subjected to a pressure in an enclosure. The mold as limit is that through which the state of internal resonance is provoked, but the mold is not that through which the internal resonance is realized; the mold is not that which, within the plastic earth, uniformly transmits in all directions the pressure and displacements. One cannot say that the mold gives form; it is the earth which takes form according to the mold, because it communicates with the workman. The positivity of this capture of form belongs to the earth and to the workman; it is this internal resonance, the work of this internal resonance.\footnote{\ At this moment, matter is no longer pre-individual matter or molecular matter, but already individual. The potential energy which actualizes itself is an inter-elementary state of the system vaster than matter.} The mold intervenes as a condition of closing, limiting, stopping of expansion, direction of mediation. The technical operation institutes the internal resonance in the matter taking form, by means of topological conditions that can be named form, and the energy conditions that express the whole system. Internal resonance is a state of the system which requires this realization of the energy conditions, the topological conditions and the material conditions: resonance is movement and energy exchange in a determined enclosure, communication between a microphysical matter and a macrophysical energy starting from a singularity of average magnitude, topologically definite.

%----------------------------------------
% Section 2
%----------------------------------------

\newpage
\section{The Validity of the Hylemorphic Model}

The technical operation of the capture of form can thus be used as a paradigm provided that one asks this operation to indicate the true relations which it institutes. However, these relations are not established between the raw material and the pure form, but between the prepared matter and materialized forms: the operation of the capture of form does not suppose only raw material and form, but also energy; the materialized form is a form that can act as a limit, as a topological border of a system. The prepared matter is that which can transport the potential energy which charges it in the technical manipulation. The pure form, playing a role in the technical operation, must become a system of points of application corresponding to the reactive forces, while the raw material becomes a homogeneous vehicle of potential energy. The capture of form is a common operation of the form and matter in a system: the condition of energy is essential, and it is not furnished by the form alone; it is the whole system that is the focus of potential energy, precisely because the capture of form is an in-depth operation throughout the entire mass, in consequence of an energy state of reciprocity of the matter in relation to itself.\footnote{\ This reciprocity causes a permanent energetic disposal: in a very limited space a considerable amount of work can effectuate itself if a singularity attracts a transformation there.} It is the distribution of the energy which is determining in the capture of form, and the mutual suitability of the matter and the form is related to the possibility of existence and the characters of this energy system. The matter is what transports this energy and the form what modulates the distribution of this same energy. The unity matter-form, at the time of the capture of form, is in the field of energy.

The hylemorphic model retains only the ends from these two half-chains that the technical operation elaborates; the schematics of the operation itself is veiled, been ignored. There is a hole in the hylemorphic representation, making the true mediation disappear, the operation itself which attaches one to the other both half-chains by instituting an energy system, a state that has evolved and must indeed exist so that an object appears with its haecceity. The hylemorphic model corresponds to the knowledge of a man who remains outside the workshop and considers only what enters there and what is done there; to know the true hylemorphic relation, it is not enough even to penetrate inside the workshop and to work with the craftsman: one would need to penetrate inside the mold itself to follow the operation of the capture of form to the various levels of the dimensions of physical reality.

Seizure in itself, the operation of the capture of form can effectuate itself in many ways, according to various methods apparently very different from each other. The true technicality of the operation of the capture of form largely exceeds the conventional limits which separate trades and the fields of work. Thus, it becomes possible, by the study of the energy field of the capture of form, to bring closer the molding of a brick to the operation of an electronic relay. In an electron tube of the triode type, the “matter” (vehicle of potential energy which actualizes itself) is the cloud of electrons leaving the cathode in the circuit cathode-anode-effector-generator. The “form” is what limits this actualization of potential energy in reserve in the generator, i.e. the electric field created by the potential difference between the grid of order and the cathode, which is opposed to the cathode-anode field, created by the generator itself; this counter-field is a limit to the actualization of the potential energy, as the walls of the mold are a limit to the actualization of the potential energy of the system clay-mold, transported by the clay in its displacement. The difference between the two cases lies in the fact that, for clay, the operation of the capture of form is finished in time: it tends, rather slowly (in a few seconds) towards a state of equilibrium, until the brick is taken from the mold; one uses the state of equilibrium while un-molding when it is reached. In the electron tube, one employs a support of energy (the cloud of electrons in a field) of a very weak inertia, so that the state of equilibrium (adequacy between the distribution of the electrons and the gradient of the electric field) is obtained in an extremely rapid time compared to the preceding (some billionths of a second in a tube of greater dimensions, some tenth of a billionth of a second in the smaller tubes).

Under these conditions, the potential of the grid of order is used as a variable mold; the distribution of the support of energy according to this mold is so fast that it is carried out within the smallest minimum time for the majority of the applications: the variable mold is then used to vary in time the actualization of the potential energy of a source; one has stopped not when equilibrium is reached, one continues by modifying the mold, i.e. the grid voltage; actualization is almost instantaneous, there is no end to its release from the mold, because the circulation of the support of energy is equivalent to a permanent release from the mold; a modulator is a continuous temporal mold. The “matter” is there almost only as the support of potential energy; it however always preserves a defined inertia, which prevents the modulator from being infinitely fast. In the case of the clay mold, that which, on the contrary, is technically used as the state of balance that one can preserve while un-molding: one then accepts a rather large viscosity of clay so that the form is conserved during the release from the mold, although this viscosity slows down the capture of form. In a modulator of energy, because one does not seek to preserve the state of balance after the conditions of equilibrium have been met: it is easier to modulate energy carried by compressed air. The mold and the modulator are extreme cases, but the essential operation of the capture of form is achieved there in the same way; it consists of the establishment of energy, durable or not. To mold is to modulate in a final way; to modulate is to mold in a continuous and perpetually variable way.

A great number of technical operations use a capture of form that has intermediate characters between the modulation and the molding; thus, a spinneret, a rolling mill, are molds in a continuous mode, creating by successive stages (master keys) a final profile; the release from the mold is continuous there, as in a modulator. One could design a rolling mill which would really modulate the matter, and would manufacture, for example, a crenulated or dented bar; rolling mills that produce corrugated sheet iron modulate the matter, while a rolling mill smooths only a model. Molding and modulation are the two borderline cases whose modeling is the average case.

We would like to show that the technological paradigm is not deprived of value, and that it is possible up to a certain point to think the genesis of individuated beings, but under the express condition that one retains as an essential model the relation of the matter in the form through the energy system of the capture of form. Matter and form must be seized during the capture of form, at the moment when the unity of the becoming of an energy system constitutes this relation on the level of the homogeneity of forces between the matter and the form. What is essential and central, is the operation of energy, supposing energy potentiality and a limit of actualization. The initiative of the genesis of substance returns neither to the raw material as passive nor to the form as pure: it is the complete system that generates, and it generates because it is a system of actualization of potential energy, joining together in an active mediation two realities, of different orders of magnitude, in an intermediate order.

Individuation, in the classical sense of the term, cannot have its principle in the matter or the form; neither form nor matter is enough with the capture of form. The true principle of individuation is the genesis itself taking place, i.e. the system in becoming, as its energy self-actualizes. The true principle of individuation can neither be sought in what exists before the individuation occurs, nor in what remains after the individuation is accomplished; it is the system of energy that is individuating insofar as it realizes in the individual this internal resonance of the matter taking form and a mediation between orders of magnitude. The principle of individuation is the single way in which the internal resonance of this matter is established taking this form. The principle of individuation is an operation. With the result that a being is itself, different from all the others; it is neither its matter nor its form, but it is the operation by which its matter took form in a certain system of internal resonance. The principle of individuation of brick is not the clay, nor the mold: this heap of clay and this mold will leave other bricks than this one, each one having its own haecceity, but it is the operation by which the clay, at a given time, in an energy system which included the finest details of the mold as the smallest components of this wet dirt took form, under such pressure, thus left again, thus diffused, thus self-actualized: a moment ago when the energy was thoroughly transmitted in all directions from each molecule to all the others, of the clay to the walls and the walls to the clay: the principle of individuation is the operation that carries out an energy exchange between the matter and the form, until the unity leads to a state of equilibrium. One could say that the principle of individuation is the common allagmatic operation of the matter and form through the actualization of potential energy.\footnote{\ The Greek word \lq{}allagma\rq{} can mean change or vicissitude, but it can also mean that which can be given or taken in exchange, which more genuinely captures the idea of energy exchange here [Tr. Note].} This energy is energy of a system; it can produce effects in all the points of the system in an equal way, it is available and is communicated. This operation rests on the singularity or the singularities of the concrete here and now; it envelops them and amplifies them.\footnote{\ These real singularities, occasions of a common operation, can be called information. The form is an apparatus for producing them.}

%----------------------------------------
% Section 3
%----------------------------------------

\newpage
\section{The Limits of the Hylemorphic Model}

However, one cannot extend in a purely analogical way the technological paradig\-m to the genesis of all beings. The technical operation is complete in a limited time; after actualization, it leaves a partially individuated, more or less stable being which draws its haecceity from this operation of individuation having constituted its genesis in a very short time; the brick, at the end of a few years or several thousand years, again becomes dust. The individuation is complete in one stroke; the individuated being is never individuated more perfectly than when it leaves the hands of the craftsman. There thus exists a certain externality of the operation of individuation compared to its result. Quite to the contrary, in the living being, the individuation is not produced by only one operation, limited by time; the living being is in itself partially its own principle of individuation; it continues its individuation, and the result of a first operation of individuation, instead of being only one result which gradually degrades, becomes the principle of a later individuation. The individuating operation and the individuated being are not in the same relation except in the product of the technical effort.

To become a living being, instead of being a becoming following individuation, is always to become between two individuations; individuating and individuated are in the living being in a prolonged allagmatic relation. In the technical object, this allagmatic relation exists only for a moment, when both half-chains are connected one to the other, i.e. when the matter takes form: in this moment, individuated and individuating are coincident; when this operation is finished, they separate; the brick does not carry its mold, and it is detached from the workman or the machine that pressed it.\footnote{\ It only manifests the singularities of the here and now constituting the conditions of information of its particular mold: state of usury of the mold (engravings, irregularities).} The living being, after being begun, continues individuating itself; as time individuates the system and partial results of individuation. A new mode of internal resonance is instituted in the living being whose technology does not provide the paradigm: a resonance through time, created by the recurrence of the results going up towards the principle and becoming the principle in its turn. As in the technical individuation, a permanent internal resonance constitutes the unity of the organism. But, moreover, with this simultaneous resonance a successive resonance is superimposed, a temporal allagmatic. The principle of individuation of the living is always an operation, like the capture of technical Form, but this operation is of two dimensions, that of simultaneity, and that of succession, through an ontogenesis supported by memory and instinct.

One can then wonder whether the true principle of individuation is not indicated better by the living than by the technical operation, and if the technical operation could be known as individuating without the implicit paradigm of the life exists in us, that knows the technical operation and practices it with our body diagram, our practices, and our memory. This question is of a wide philosophical range, because it results in wondering whether a true individuation can exist apart from life. For knowledge, it is not the technical, anthropomorphic and consequently zoomorphic operation that is necessary to study, but the natural processes of formation of the basic unities that nature presents apart from the domain defined as the living.

Thus, the hylemorphic model, departing from technology, is insufficient under its usual species, because it is even unaware of the center of the technical operation of the capture of form, and led in this direction to be unaware of the role played by the conditions of energy in the capture of form. Moreover, even restored and completed in the form of the triad matter-form-energy, the hylemorphic model is likely to wrongly objectify a contribution of the living in the technical operation; it is this fabricated intention which constitutes the system thanks to which the energy exchange is established between matter and energy in the capture of form; this system does not form part of the individuated object; however, the individuated object is thought by mankind as having an individuality as a manufactured object, by reference to the manufacture. The haecceity of this brick as brick is not an absolute haecceity, it is not the haecceity of this preexistent object due to the fact that it is a brick. It is the haecceity of the object as a brick: it comprises a reference for use and, through it, to the fabricated intention, therefore with the human gesture which constituted the two half-chains joined together in a system for the operation of the capture of form.\footnote{\ The individuality of the brick, that by what this brick expresses such operation that have existed here and now, envelops the singularities of this here and now, prolongs them, amplifies them; however, the technical production seeks to reduce the margin of variability, of unpredictability. The real information that modulates an individual seems like a parasite; it is that by which the technical object remains in some measurement inevitably natural.}

In this sense, the hylemorphic model is perhaps only apparently technological: it is the reflection of the vital processes in an abstractly known operation and draws its consistency of what it is made by a living being for living beings. This would explain the very great paradigmatic capacity of the hylemorphic model: coming from the living, it goes back there and applies to it, but with a deficiency owing to the fact that the awakening which has clarified it seizes it through the wrongly simplified particular case of the technical capture of form; it seizes types more than individuals, specimens of a model more than of realities. The dualism matter-form, seizing only the extreme terms of that which is larger and smaller than the individual, obscures the reality that is of the same order of magnitude that produced the individual, and without which the extreme terms would remain separate: an allagmatic operation spreading itself starting from a singularity.

However, it is not enough to criticize the hylemorphic model and to restore a more exact relation in the course of the technical capture of form to discover the true principle of individuation. It is also not enough to suppose in the knowledge that one takes from the technical operation a paradigm initially biological: even if the relation matter-form in the technical capture of form is easily known (adequately or inadequately) thanks to the fact that we are living beings, it is not more important than the reference to the technical field that makes it necessary for us to clarify, explicate, and objectify this implicit concept that the subject carries with it. If testing the vital is the condition of the represented technique, the represented technique becomes in its turn the condition of the knowledge of the vital. One is thus returned from one order to another, so that the hylemorphic model seems to owe its universality mainly to the fact that it institutes reciprocity between the vital domain and the technical field. Besides, the model is not the only example of a similar correlation: the automatism to penetrate the functions of the living by means of representations resulting from technology, from Descartes to current cybernetics. However, an important difficulty emerges in the hylemorphic use of the model: it does not indicate what is the principle of individuation of the living, precisely because it grants to the two terms an existence prior to the relation which links them, or at least because it cannot make it possible to think this relation clearly; it can represent only the mixture, or attachment part by part; the way in which the form informs the matter is not enough for the hylemorphic model. To use the hylemorphic model is to suppose that the principle of individuation is in the form or in the matter, but not in the relation of both. The dualism of substances–soul and body–is in the seed of the hylemorphic model, and one can wonder whether this dualism will leave the technique in good condition.

In order to look further into this examination, it is necessary to consider all the conditions that surround a notional capture of consciousness. If there were only the living individual being and the technical operation, the hylemorphic model perhaps could not be constituted. In fact, it seems well that the middle term between the living field and the technical field was, at the hylemorphic origin of the model, social life. What the hylemorphic model reflects initially is a socialized representation of work and a representation also socialized of the individual living being; the coincidence between these two representations is the foundation common to the extension of the diagram from one field to the other, and the guarantor of its validity in a given culture. The technical operation which imposes a form on a passive and unspecified matter is not only an operation considered abstractly by the spectator who sees between the workshop and what is produced without knowing the development properly stated. It is primarily the operation commanded by the free man and executed by the slave; the free man chooses matter, unspecified because it is generically enough to the designer under the name of substance, without seeing it, without handling it, without preparing it: the object will be made of wood, or iron, or out of the earth. Truthfully, the passivity of matter is its availability abstracted behind the given order that others will carry out. Passivity is that of the human mediation which will retrieve the matter. The form corresponds to that which the man who commands has thought by himself and which he must express in a positive way to whom he gives his orders: the form is thus of the order of the expressible; it is eminently active because it is what one imposes on those who will handle the matter; it is the same contents of the order, that through which it governs. The active character of the form and the passive character of the matter answer the conditions of the transmission of the order which supposes social hierarchy: it is in the contents of the order that the indication of matter is undetermined and at the same time form is determination, expressible and logical. It is through social conditioning that the soul is opposed to the body; it is not through the body that the individual is citizen, participating in collective judgements, common beliefs, surviving in the memory of his fellow citizens: the soul is distinguished from the body as the citizen from the human living being. The distinction between form and matter, the soul and the body, reflects a city that contains citizens in opposition to the slaves. One must notice however that the two designs, technological and civic, if the citizens agree to distinguish the two terms, do not assign to them the same role in the two couples: the soul is not pure activity, full determination, whereas the body would be passivity and indetermination. The citizen is individuated as a body, but he or she is also individuated as a soul.

The vicissitudes of the hylemorphic model owes to the fact that it is neither directly technological nor directly vital: it is a technological operation and a vital reality mediated by the social, i.e. by the conditions already given—in inter-individual communication—from an effective reception of information, in the species the order of fabrication. This communication between two social realities, this operation of reception which is the condition of the technical operation, masks what, within the technical operation, allows two extreme terms—form and matter—to enter into interactive communication: information, the singularity of the “here and now” of the operation, pure event in the dimension of the appearing individual.

%----------------------------------------
% Back Page
%----------------------------------------

\newpage

\vspace*{\fill}

\begin{center}

This work was republished with permission of the author.\\
[1\baselineskip]
For any queries contact Alexander Longworth-Dunbar at:\\ \href{mailto:alexander.longworth-dunbar@student.manchester.ac.uk}{alexander.longworth-dunbar@student.manchester.ac.uk}

\end{center}

\end{document}

%----------------------------------------
% End of Document
%----------------------------------------